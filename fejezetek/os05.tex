% Reguláris kifejezések

Sok program (főleg szűrők) használ \emph{mintaillesztést} (pattern matching), \emph{mintakeresést} (pattern scanning) és mintafeldolgozást (pattern processing). Ilyen esetekben a -- legtöbbször szöveges -- bemeneti adatok azon részével fog dolgozni a program, amely egy megadott mintának megfelel, azaz a mintára illeszkedik (vagy amire a minta illeszkedik). Az ilyen komplex minták egyik gyakran alkalmazott formája a szabályos avagy \emph{reguláris kifejezés} (regular expression, regexp, RE).

A reguláris kifejezésekkel mélyebben a formális nyelvek elmélete (theory of formal languages) foglalkozik.

\bigskip

\textbf{Leírás:} man 7 regex, man grep, info grep, man awk/gawk, info gawk

\bigskip

Egy reguláris kifejezés a szövegnek mindig a legkorábban elkezdődő, és ezen belül a leghosszabb részére illeszkedik. Ez a részkifejezésekre is igaz. Az illeszkedő rész a szövegen belül bárhol – akár egy szó belsejében is – előfordulhat, kivéve néhány esetet (pl. %\^ és \$, ld. következő dia).

Alapesetben a kisbetűk és nagybetűk különbözőnek számítanak illesztéskor.

A reguláris kifejezésekben néhány karakternek speciális jelentése van. Mivel ezek közül sokat a shell is speciálisan kezel, így a parancssorban megadott reguláris kifejezést érdemes aposztrófok közé zárni.

\subsection{Elemi kifejezések}

\begin{description}
 \item[(KIF)] csoportosítás (műveleti sorrend felülbírálása), \texttt{KIF}-re illeszkedik
 \item[{()}] az üres szóra illeszkedik
 \item[ [\textbf{HALMAZ}] \hspace{-3mm} ]\hspace{+3mm}  A halmaz bármely karakterének egy példányára illeszkedik. A halmazt a karakterek egymás mellé írásával adhatjuk meg.
\item[ [\textbf{TOL-IG}] \hspace{-3mm} ]\hspace{+3mm}  mint előbb, de itt egy tartományt adunk meg
\item[ [\textasciicircum \textbf{HALMAZ}] \hspace{-3mm} ]\hspace{+3mm}  a halmazban nem szereplő bármely karakter egy példányára illeszkedik (a sortörést kivéve)
\item[ . ] bármilyen karakter egy példányára illeszkedik (a sortörést kivéve)
\item[ \textasciicircum ] a sor elejére illeszkedik
\item[\$] a sor végére illeszkedik
\item[$\backslash$KARAKTER] a $\backslash$ után írt speciális jelentésű karaktert közönségesként kezeli
\item[KARAKTER] bármely közönséges karakter saját maga egy példányára illeszkedik
\end{description}

\subsection{Összetett kifejezések}
\begin{description}
\item[$\mathbf{KIF_1KIF_2}$] (két kifejezés egymás mellé írása): Összefűzés, konkatenáció (concatenation). Olyan szövegre illeszkedik, amelynek első fele KIF1-re, második fele KIF2-re illeszkedik. Több kifejezést is összefűzhetünk.
\item[$\mathbf{KIF_1|KIF_2|\dots}$] Logikai MEGENGEDŐ VAGY (diszjunkció), alternáció (alternation). 
	Olyan szövegre il\-lesz\-ke\-dik, amely legalább az egyik kifejezésre (alternatívára) illeszkedik.
ismételt illesztés, ismétlésszám megadása, iteráció (repetition, iteration):
\item[KIF*] KIF akárhány egymást követő példányára illeszkedik (0 is)
\item[KIF+] KIF legalább 1 egymást követő példányára illeszkedik
\item[KIF?] KIF 0 vagy 1 példányára illeszkedik (azaz KIF opcionális)
\item[KIF\{I\}] KIF pontosan I egymást követő példányára illeszkedik
\item[KIF\{I,\}] KIF legalább I egymást követő példányára illeszkedik
\item[KIF\{I,J\}] mint előbb, de legfeljebb J példányra illeszkedik ($I\leq J$)
\end{description}

Műveleti erősség csökkenő sorrendben: iteráció, konkatenáció, alternáció




\subsection{grep (ismét)}
\begin{center}
\verb@grep ’REGKIF’ ÁLLOMÁNY(OK) @
\end{center}

Kiírja a megadott állomány(ok) mindazon sorait, amelyek illeszkednek a REGKIF reguláris kifejezésre. Szűrőnek tekinthető.
%
A korábban említettek miatt az aposztrófok kiírása ajánlott.
%
Ha nem adunk meg állományt, akkor a szabványos bemenetről olvas.
\bigskip

\textit{Kapcsolók}
\begin{description}
 \item[-c] Az illeszkedő sorok tartalma helyett csak azok darabszáma jelenik meg. A -v opció esetén a nem illeszkedő sorok száma íródik ki.
\item[-E] Teljes értékű, kibővített (extended) kifejezések használata. Ha ezt elhagyjuk, akkor a reguláris kifejezéseknek egy régebbi változatát kell megadnunk. Ez utóbbi jelentősen eltér a korábban bemutatottól!
\item[-e ’REGKIF’] Akkor kell használni, ha a reguláris kifejezés - jellel kezdődik. Közvetlenül a REGKIF előtt kell állnia!
\item[-F] REGKIF-ben minden karaktert közönségesként értelmez
\item[-f \texttt{KIFFÁJL}] A \texttt{KIFFÁJL} minden sorát egy-egy REGKIF-nek tekinti. Ilyenkor a bármelyik kifejezésre illeszkedő sorok jelennek meg
\item[-i] a kisbetűket és a nagybetűket azonosnak tekinti
\item[-n] az illeszkedő sorok tartalma elé a sorszámukat is kiírja
\item[-o] a sorokból csak az illeszkedő részt jeleníti meg
\item[-R, -r] Ha könyvtárat adtunk meg, akkor a keresés az alkönyvárakban és azok teljes tartalmában történik (rekurzív keresés).
\item[-v] illeszkedés helyett nem-illeszkedést vizsgál (inverzió)
\item[-w] Csak olyan sort ír ki, amelyben legalább egy egész szó (nemcsak egy részlet) illeszkedik a reguláris kifejezésre
\end{description}

A gyakorlatban a \texttt{-E} opció vagy a vele ekvivalens \texttt{egrep} parancs használata ajánlott! 
Ha nem így tennénk, vegyük figyelembe, hogy ezek nélkül a reguláris kifejezéseknek egy régebbi 
(basic) változatát kell használnunk, ahol pl. a ( és ) közönséges karakterek, a csoportosításra pedig a 
$\backslash($ és $\backslash)$ jelölések szolgálnak (tehát a korábban látotthoz képest pont fordítva működnek).
 Ugyanez érvényes a $\{, \},$ |, ? és + karakterekre is.

\subsection{Példák}
\begin{itemize}
 \item \textbf{alma} azt jelenti, hogy a minta a soron belül bárhol előfordulhat
 \item \textbf{\textasciicircum alma} előírja, hogy a mintának a sor elején kell előfordulnia
\item \textbf{\textasciicircum [mh]?alma} azt jelenti, hogy a sor elején alma, malma vagy halma mintának kell előfordulnia
\item \textbf{\textasciicircum [\textasciicircum mh]alma}
  azokra a sorokra illeszkedik, melyek elején nem szerepel az alma, malma vagy halma sorozat
% \item \textbf{}
% \item \textbf{}
\end{itemize}

\begin{lstlisting}
$ ls -l | grep ^d
\end{lstlisting}
kiírja a munkakönyvtárban levõ összes katalógust

\begin{lstlisting}
$ grep '^main' *.c 
\end{lstlisting}
kiírja a .c végű állományok azon sorait, amelyek a main karaktersorral kezdődnek.

\subsection*{Feladatok}
\begin{enumerate}
\item Írjunk reguláris kifejezést, ami az egész számokra illeszkedik!
	

\item Írjunk reguláris kifejezést, ami a szabályos e-mail címekre illeszkedik!

	Tegyük fel, hogy e-mail cím általános alakja: \texttt{local@domain}, ahol a 	
	\begin{description}
	\item[local] az angol abc kisbetűit, számokat és kötőjelet, pontot illetve aláhúzást (\_) tartalmazhat, de a\-lá\-hú\-zás\-sal nem kezdődhet
	\item[domain] részben az abc kisbetűi, számok és kötőjel, illetve pontok lehetnek. 
	%	^[_a-z0-9-]+(\.[_a-z0-9-]+)*@[a-z0-9-]+(\.[a-z0-9-]+)*$
	\end{description}

	A fenti egy egyszerűsített leírása a ténylegesen érvényes e-mail címeknek. 
	Nem lehet például pont a @ előtt, illetve nem lehet két pont egymás után (sem a local, sem a domain részben)\footnote{További infó: \url{http://en.wikipedia.org/wiki/Email\_address\#Syntax}}.

\item Próbáljunk szabályos magyar címekre szűrni, ahol tegyük fel, hogy a cím az alábbihoz hasonló formában fordulhat elő:

	\texttt{XXXX. Város, Közterületnév utca/út/körút/sétány/tér/park szám.}

	Figyeljünk arra, hogy közterület neve állhat több szóból is!
	
\end{enumerate}
