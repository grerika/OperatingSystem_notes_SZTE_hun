\subsection{if}

\begin{minipage}{0.425\textwidth}
\lstset{linewidth=\textwidth}
\begin{lstlisting}
if utasitas; then
    utasitasblokk
fi
\end{lstlisting}
\end{minipage}
\hspace{1em}
\begin{minipage}{0.425\textwidth}
\lstset{linewidth=\textwidth}
\begin{lstlisting}
if utasitas
then
    utasitasblokk
fi
\end{lstlisting}
\end{minipage}


\noindent 
A BASH végrehajtja az utasítást és ha igaz (0 a visszatérési értéke), 
  lefuttatja az utasításblokk utasításait is.
\begin{lstlisting}
if utasitas; then
    utasitasblokk1
else
    utasitasblokk2
fi
\end{lstlisting}

\noindent
A BASH az utasítást végrehajtja: Ha igaz, az igaz ág, ha hamis (nem 0),
a hamis ág utasításait hajtja végre.\medskip

A feltételként szereplő utasítást pontos vessző, vagy újsor karakter zárja le.\bigskip

\emph{Példa}
\begin{lstlisting}
#! /bin/bash

if mount /mnt/floppy; then
    echo "Beillesztes megtortent."
else
    echo "Beillesztes sikertelen."
fi
\end{lstlisting}

\subsection{Logikai műveletek}
Programok visszatérési értékein használhatjuk az alábbi logikai operátorokat.

\begin{table}[!h]
\centering
\begin{tabular}{|cl|}
\hline
\tt!	& tagadás(prefix)\\
\tt|| 	& logikai vagy(infix)\\
\tt\&\&	& logikai és(infix)\\
\hline
\end{tabular}
\caption{Logikai műveletek}
\end{table}


\bigskip
\begin{minipage}{0.425\textwidth}

\subsubsection*{Tagadás}
\lstset{linewidth=\textwidth}
\begin{lstlisting}
#! /bin/bash

if ! mkdir $HOME/tmp2; then
  echo "Nem sikerult letrehozni."
fi
\end{lstlisting}
\end{minipage}
\hspace{1em}
\begin{minipage}{0.425\textwidth}
\lstset{linewidth=\textwidth}
\subsubsection*{ÉS}
\begin{lstlisting}
#! /bin/bash

if mkdir tmp && cp ak tmp; then
    echo "Sikerult!"
fi
\end{lstlisting}
\end{minipage}
\subsubsection*{VAGY}
\begin{lstlisting}
#! /bin/bash

if mkdir tmp || mkdir tmp2; then
    echo "Legalabb egyik konyvtar letrejott!"
fi
\end{lstlisting}

\noindent Utasításokból logikai operátorok segítségével képzett kifejezések szintén utasítások, de elsőre úgy tűnhet, hogy talán kissé különösen viselkednek.
\vfill\pagebreak

\subsubsection*{Lusta kiértékelés mellékhatásai}
Az alábbi esetben, ha az első utasítás értéke hamis, az ÉS művelet miatt a végeredmény mindenképp hamis lesz, felesleges tehát futtatni a második utasítást!
\begin{lstlisting}
      utasitas1 && utasitas2
\end{lstlisting}


\noindent Az alábbi esetben pedig ha az első utasítás értéke igaz, a VAGY művelet miatt a végeredmény mindenképp igaz lesz, felesleges tehát futtatni a második utasítást!
\begin{lstlisting}
      utasitas1 || utasitas2
\end{lstlisting}


\emph{Példa}
\begin{lstlisting}
#!/bin/bash

ls -ld ak || echo "Nem erheto el!"
\end{lstlisting}
Az ilyen szerkezetek úgy működnek, mint az if szerkezet: a második utasítás végrehajtása az 
első eredményétől függően következik be.

\subsection{Logikai állandók}
Logikai állandót a \verb.true. és \verb.false. programok segítségével használhatunk a BASH-ban.
\begin{description}
\item[true] A program nem csinál semmit, visszatérési értéke igaz.
\item[false] A program nem csinál semmit, visszatérési értéke hamis.
\end{description}


\subsection{A test program}
Az aritmetikai relációk kiértékelésére és az állományvizsgálatra általában a \texttt{test} parancsot használjuk.

\begin{minipage}{0.42\textwidth}
\lstset{linewidth=\textwidth}
\begin{lstlisting}[rulecolor=\color{red},rulesepcolor=\color{red}]
 #!/bin/bash

if 8<9;then
    echo "Nyolc kisebb, mint kilenc."
fi
\end{lstlisting}
\end{minipage}
\hspace{1em}
%\hfill
\begin{minipage}{0.42\textwidth}
\lstset{linewidth=\textwidth}
\begin{lstlisting}[rulecolor=\color{red},rulesepcolor=\color{red}]
#!/bin/bash

if 8 < 9;then
    echo "Nyolc kisebb, mint kilenc."
fi
\end{lstlisting}
\end{minipage}

\noindent{\color{red}\textbf{Nagyon rossz!}} Ez azt jelenti, hogy a 8 nevű program szabványos bemenetére irányítsuk a 9 nevű állományt!
\bigskip

\subsubsection*{Helyesen}

\begin{lstlisting}[rulecolor=\color{green},rulesepcolor=\color{green}]
 #!/bin/bash

if test 8 -lt 9; then
    echo "Nyolc kisebb, mint kilenc."
fi
\end{lstlisting}

\subsubsection*{A test kettős viselkedése}

\begin{minipage}{0.425\textwidth}
\lstset{linewidth=\textwidth}
\begin{lstlisting}
#! /bin/bash

if test 8 -lt 9; then
    echo "Nyolc kisebb, mint kilenc."
fi
\end{lstlisting}
\end{minipage}
\hspace{1em}
\begin{minipage}{0.425\textwidth}
\lstset{linewidth=\textwidth}
\begin{lstlisting}
#! /bin/bash

if [ 8 -lt 9 ]
then echo "Nyolc kisebb, mint kilenc."
fi
\end{lstlisting}
\end{minipage}

\noindent FONTOS: a jobb oldali szintakszis esetén, a \verb.[. után illetve a \verb.]. előtt mindenképp használnunk kell szóközöket!



\subsubsection*{Fájlvizsgálat}
\begin{lstlisting}
#!/bin/bash
if [ ! -d $HOME/bin ]; then
    mkdir $HOME/bin
fi 
\end{lstlisting}


\subsubsection*{Karakterlánc példa}
\begin{lstlisting}
#!/bin/bash

echo -n "Konyvtarnev [bin]:"
read KVT
if [ -z "$KVT" ];then
    KVT=bin
fi
\end{lstlisting}
\emph{Megjegyzés:} ha a felhasználó nem ad meg karakterláncot és az idézőjeleket nem használjuk, akkor innen
hiányzik a paraméter!


% \subsubsection*{Fájlvizsgálat}
\begin{table}[!h]
\centering
 \begin{tabular}{|ll|}
          \hline
\tt -b fájlnév	& blokkeszköz-meghajtó
\\
\tt -c fájlnév	& karaktereszköz-meghajtó
\\
\tt -d fájlnév	& könyvtár
\\
\tt -f fájlnév	& szabályos állomány
\\
\tt -L fájlnév	& közvetett hivatkozás
\\
\tt -p fájlnév	& csővezeték
\\
\tt -e fájlnév	& létezik
\\
\tt -G fájlnév	& saját csoportba tartozik
\\
\tt -O fájlnév	& saját tulajdon
\\	
\tt -r fájlnév	& olvasható
\\
\tt -w fájlnév	& írható
\\
\tt -x fájlnév	& futtatható
\\
\tt -s fájlnév	& nem üres
\\
\hline
\tt fájl1 -nt fájl2	& a fájl1 újabb, mint a fájl2
\\
\tt fájl1 -ot fájl2	& a fájl1 régebbi, mint a fájl2
\\
\tt fájl1 -ef fájl2	& a fájl1 és fájl2 azonos állományt jelöl
\\
\hline
  \end{tabular}
\caption{Fájlvizsgálat a test paranccsal}
 \end{table}

\begin{table}[!ht]
\centering
 \begin{tabular}{|ll|}
          \hline
\tt Kif1 -eq Kif2	& Egyenlő
\\
\tt Kif1 -ne Kif 2	& Nem egyenlő
\\
\tt Kif1 -lt Kif2	& Kisebb
\\
\tt Kif1 -le Kif2	& Kisebb vagy egyenlő
\\
\tt Kif1 -gt Kif2	& Nagyobb
\\
\tt Kif1 -ge Kif2	& Nagyobb vagy egyenlő
\\
\hline
\tt Kif1 -a Kif2	& Logikai és
\\
\tt Kif1 -o Kif2	& Logikai vagy
\\
\tt !Kif		& Logikai tagadás
\\
\hline
  \end{tabular}
\caption{Numerikus és logikai operátorok}
% \end{table}
\medskip
% \begin{table}
\centering
 \begin{tabular}{|ll|}
          \hline
\tt -z String		& 0 hosszúságú
\\
\tt String			& nem 0 hosszúságú
\\
\tt String != String	& nem egyenlők
\\
\tt String = String		& egyenlők
\\
\hline
  \end{tabular}
\caption{Numerikus és logikai operátorok}
\end{table}

\vfill\pagebreak

\subsection{while}

\begin{lstlisting}
while utasitas; do
    utasitasblokk
done
\end{lstlisting}

Amíg a for szerkezet esetében lehetetlen volt, addig a while szerkezet esetében lehetséges az
utasítások végtelen ismétlése, a végtelen ciklus. A végtelen ciklusba került programot a
\Ctrl+\ c billentyűkombinációval vagy a kill programmal szakíthatjuk meg.\bigskip


\begin{lstlisting}
#! /bin/bash
while true; do
    echo "Futok..."
    sleep 3
done
\end{lstlisting}
A \texttt{sleep} parancs felfüggeszti a program futását a paramétere által megadott másodpercre.



\begin{lstlisting}
#!/bin/bash
SZOVEG="You have a new message."
while [ ! -s "$MAIL" ]; do
    sleep 3
done
echo $SZOVEG
play /usr/share/sounds/email.wav
echo $SZOVEG | festival --tts
\end{lstlisting}

Végtelen ciklus segítségével
\begin{lstlisting}
#!/bin/bash
SZOVEG="You have a new message."
while true;do
    while [ ! -s "$MAIL" ];do
         sleep 3
    done
    echo $SZOVEG
    play /usr/share/sounds/email.wav
    echo $SZOVEG | festival --tts
    while [ -s "$MAIL" ];do
         sleep 3
    done
done
\end{lstlisting}


\subsection{do-until}
\begin{lstlisting}
until utasitas; do	
    utasitasblokk
done
\end{lstlisting}

Az until szerkezet addig ismétli az utasításblokkot, amíg az utasítás visszatérési értéke hamis. 
Az until tehát while szerkezethez képest a feltételként adott utasítás értelmét az ellentettjére változtatja.

\vfill\pagebreak
\subsection{case}
 \begin{lstlisting}
case szo in
    minta1a|minta1b)
          utasitasblokk1;;
    minta2)
          utasitasblokk2;;
    *) utasitasblokk0
esac
 \end{lstlisting}

A case szerkezet illeszti a szót a megadott mintákra, majd azt az utasításblokkot hajtja végre,
amelyik az illeszkedő minták közül az elsőhöz tartozik. A case a minták kezelésénél az
állománynévhelyettesítőkarakterek kezelésének szabályait használja.
\bigskip

\emph{Példák}

\begin{minipage}{0.425\textwidth}
Példa: \texttt{scriptek/bash/case.sh}
\lstset{linewidth=\textwidth}
\begin{lstlisting}
 #!/bin/bash

read K

case $K in
    *.jpeg)
        mv $K $(basename $Kjpeg)jpg;;
    *.gif)
        convert $K $(basename $Kgif)jpg
esac
\end{lstlisting}
\end{minipage}
\hspace{1em}
\begin{minipage}{0.425\textwidth}
\lstset{linewidth=\textwidth}
\begin{lstlisting}
 #!/bin/bash

# Peldaprogram a BASH verziojanak megallapitasara
# Forras: 
# http://conectiva.com.br/~aurelio/programas/bash/txt2regex


case "$BASH_VERSION" in
     4.0[4-9]*|4.[1-9]*): ;;
     *)echo "bash version >=4.04 required..."; exit 1 ;;
esac
\end{lstlisting}
\end{minipage}


\subsection{Vágókifejezések}
\emph{A vágókifejezések nem kötelező anyagrész, de hasznos eszközök bash programozáskor.}

%\begin{table}[!h]
\begin{center}
 \begin{tabular}{|ll|}
\hline
\verb.${STRING#minta}.	& Ha a minta illeszkedik a string elejére, \\
			&  akkor levágásra kerül a legrövidebb illeszkedő rész
\\
\hline
\verb.${STRING##minta}.	& Ha a minta illeszkedik a string elejére, \\
			&  akkor levágásra kerül a leghosszabb illeszkedő rész
\\
\hline
\verb.${STRING%minta}. 	& Ha a minta illeszkedik a string végére, \\
			&  akkor levágásra kerül a legrövidebb illeszkedő rész
\\
\hline
\verb.${STRING%%minta}.	&Ha a minta illeszkedik a string végére,\\
			&	akkor levágásra kerül a leghosszabb illeszkedő rész
\\
\hline  
 \end{tabular}
%\caption{BASH vágókifejezések}
\end{center}
%\end{table}
\begin{lstlisting}[title=scriptek/bash/vago.sh]
#! /bin/bash
# Forras: Pere Laszlo, Linux felhasznaloi ismeretek I.
# PARAMETEREK SZAMANAK ELLENORZESE
if [ $# -ne 2 ]; then
	echo "`basename $0` - Allomany kiterjesztesenek csereje"
	echo "Hasznalat:"
	echo "		`basename $0` <regi kit> <uj kit>"
	exit 1
fi
# TENYLEGES MUNKA
for FILES in *.$1; do
	echo -n "$FILES -> ${FILES%$1}$2"
	if [ -e "${FILES%$1}$2" ]; then		
		echo " Hiba: A ${FILES%$1}$2 mar letezik"
	else
		echo
		mv $FILES ${FILES%$1}$2
	fi
done
\end{lstlisting}


\subsection{Függvények}

\begin{lstlisting}
#! /bin/bash

function fuggvenynev(){
  utasitas1
  ...
}

fuggvenynev [parameterek] # hasznalat
\end{lstlisting}

A függvények paramétereire -- hasonlóan a héjprogramokhoz -- a \verb@$1, $2, ... @ kifejezésekkel hivat\-koz\-ha\-tunk.
A függvények hozzáférnek valamennyi a főprogramban meghatározott változóhoz és maguk is lét\-re\-hoz\-hat\-nak újakat. 
Ez utóbbiakat a főprogram is látni fogja, azok nem lokálisak a függvényre nézve. 


\subsubsection*{Lokális változók, rekurzió}
Függvényekben létrehozhatunk lokális változókat is, a \texttt{local} kulcsszó segítségével.

%Például:


\begin{lstlisting}
#! /bin/bash

function faktorialis ()
{
    local n=$1;

    if [ $n = 0 ]; then
        echo 1
        return ;
    fi;

    echo $(( $n * $(faktorialis $(( $n - 1 )) ) ))
}

for n in $(seq 1 20); do
    echo "$n! = " $(faktorialis $n)
done
\end{lstlisting}


