\subsection{Alapok}
A BASH héj, mint a legtöbb héj, nemcsak egy felhasználói felület, de kifinomult, magasszintű programozási nyelvet megvalósító értelmező program (interpreter) is. A BASH ezzel a nyelvvel alkalmas a napi feladataink automatizálására, a munkakörnyezet bővítésére, testreszabására.
\bigskip

	\emph{Miért kell megismernünk?}
	\begin{itemize}
	\item Az összetettebb feladatokat akkor tudjuk elvégezni, ha a megfelelő vezérlő szerkezeteket ismerjük.
	\item Az automatizálás igen fontos eleme a számítógéphasználatnak.
	\item A Unix rendszerek felépítésében komoly szerepet kapnak a héjprogramok (glue).
	\end{itemize}


	\emph{Az értelmező}
	\begin{itemize}
	\item BASH programok szöveges állományok, amelyek futtatását rendszermag végzi
	bash program segítségével.
	%\item Kezdetben Unix rendszerek minden szöveges állományt \verb./bin/sh. segítségével futtattak.
	%\item A legtöbb Linux rendszeren \verb./bin/sh. \verb./bin/bash.
	\item Amikor elterjedten kezdtek több héjprogramot használni, szükségessé vált az
	értelmezőprogram meg\-ha\-tá\-ro\-zá\-sa. Ha a fájl első két karaktere \verb.#!. a mag az utána 
	 következő programnevet használja futtatásra.
	\end{itemize}

\subsubsection*{Változók}
Mint minden parancsokra épülő programozási nyelv, a BASH nyelve is rendelkezik változókkal. BASH változókat a parancssorban is használhatunk,
de programokban mindenképpen szükségünk van rájuk. \smallskip

	\emph{A változók}
	\begin{itemize}
	\item névvel és értékkel rendelkező eszközök, általában szöveges érték tárolására használjuk
	\item a hagyomány szerint nagybetűkkel írjuk a nevüket
	\end{itemize}

\subsubsection*{Az értékadás}
\begin{verbatim}
	változónév=kifejezés
\end{verbatim}
Az értékadás baloldalán egy változó neve, jobboldalán egy kifejezés áll. Az értékadás hatására a változó értéke felveszi a kifejezés
értékét. Az egyenlőségjel bal és jobb oldalán nem lehet szóköz!
\bigskip

%\emph{Példa}
\begin{lstlisting}
#!/bin/bash

KONYVTAR=tmp
SZOVEG="hello vilag"
\end{lstlisting}

Ha a szöveges értékben szóköz szerepel, a szöveget kettős idéző jelek (") közé kell zárnunk. Ha ezt nem tesszük meg, a BASH a szóközök mentén szétválasztja a szöveget és listaként kezeli.


\subsubsection*{Behelyettesítés}
\begin{verbatim}
...$változónév...
\end{verbatim}
A változóbehelyettesítés használatakor a változónév elé egy \$ karaktert írunk. Az adott helyre a BASH a változó értékét helyettesíti be.

%\emph{Példa}
\begin{lstlisting}
#!/bin/bash

KVT=tmp
ALKVT=elso
ALALKVT=masodik
mkdir -p $KVT/$ALKVT/$ALALKVT
\end{lstlisting}

\begin{description}
\item[\tt\$VALTOZO] A megadott nevű környezeti változó aktuális értékének behelyettesítése. Ha a változó nem létezik, üres szót kapunk.
\item[\tt\$\{VALTOZO\}] Hatása megegyezik az előzővel, de ez akkor is használható, ha közvetlenül a kifejezés után betű, számjegy vagy aláhúzásjel 
	áll (máskülönben azt a név részének tekintené a shell).
\item[\tt\$\{VALTOZO:+ERTEK\}]  \emph{alternatív érték használata} \hfill

	Ha a VALTOZO üres (nulla hosszúságú) vagy nem kapott még értéket, akkor nem történik semmi, különben az ERTEK-kel helyettesitődik a fenti kifejezés.

\item[\tt\$\{VALTOZO:-ERTEK\}] \emph{alapértelmezett érték használata}\hfill
	
	Ha \verb.VALTOZO. üres, a kifejezést \verb.ERTEK.kel helyettesíti. Különben a kifejezés értéke \verb.$VALTOZO. 
	A \verb.VALTOZO. értéke változatlan marad.

\item[\tt\$\{VALTOZO:=ERTEK\}] \emph{alapértelmezett érték hozzárendelése}\hfill
	
	Ha \verb.VALTOZO. üres, a változó értékét \verb.ERTEK.re állítja. 

\item[\tt\$\{!VALT\}] \emph{indirekció}\hfill

	A \verb.$VALT. változónevű változó értékét kéri le, tehát a \verb.!VALT. helyére  a \verb.VALT. értéke kerül.

\begin{lstlisting}
#!/bin/bash

STR="A"
A="Hello"
echo "${!STR}"
echo "$A"
\end{lstlisting}

\item[\tt\$\{VALTOZO:?UZENET\}] Ha a \verb.VALTOZO.nak nincs értéke, akkor megszakad a futás az \verb.UZENET.ben 
    megadott szöveg kiírása után
\end{description}

Futtassuk a \texttt{scriptek/bash/default.sh} szkriptet!\footnote{Megtalálható kitömörítés után a \url{http://www.inf.u-szeged.hu/~grerika/teaching/os/scriptek.tar.gz} fájlban, kitömörítéshez segítség a \pageref{subsec:archiv}. oldalon}


%A programindítása előtt a BASH változók értékét behelyettesíti a megfelelő helyekre és a behelyettesítéssel kapott értékekkel indítja a programot.

%\begin{lstlisting}
%#!/bin/bash
%NEV=Gizi
%echo Nev: $NEV
%\end{lstlisting}

%Az \texttt{echo} parancs egyszerűen kiírja az összes paraméterét a szabványoskimenetre. A kiírt listában a paramétereket egy szóköz választja el.

\subsubsection*{Érték beolvasása}
\begin{verbatim}
     read változónév
\end{verbatim}
A read utasítás a felhasználó által írt szöveget helyezi el az utána írt változóban. A felhasználó a szöveg beírása közben használhatja a szerkesztő billentyűket és a szövegben tetszőleges karaktereket elhelyezhet.


\begin{lstlisting}
#! /bin/bash
read KVT
mkdir "$KVT"
\end{lstlisting}
Az idézőjelek (") miatt a parancs akkor is egy paramétert kap, ha a felhasználó által beírt szövegben szóköz van.






\subsubsection*{Idézőjelek hatása}
A \emph{kettős idézőjelek} (\verb.". [\Alt+\ 2]) részleges elszigetelést jelölnek, ami azt jelenti, hogy a héj értelmezi és behelyettesíti a benne található, számára értelmes karaktereket. Ha például egy * karaktert talál, akkor azt a munkakönyvtárban levő fájlok neveinek listájával fogja helyettesíteni. Ha \verb.$. karaktert ,,lát'', a közvetlenül utána következő szót változónévnek vagy parancssori paraméternek fogja tekinteni és a megfelelő értéket behelyettesíti.

A \emph{szimpla (egyszeres) idézőjelek} (\verb.'. [\Alt+\ 1]) használata teljes elzárást jelent. Az ilyen karakterláncokban a héj semmiféle értelmezést nem végez, azok tartalmát betű szerint kezeli.


A \emph{visszafelé hajló} (\texttt{`} [\Alt+\ 7]) idézőjelek kifejezetten parancsvégrehajtást jelölnek. Az ilyen szöveget a héj parancssornak tekinti, végrehajtja és az eredménnyel helyettesíti. Ez a parancsbehelyettesítésnek nevezett eljárás lehetővé teszi, hogy egy változónak azonnal átadjuk egy esetleg meglehetősen bonyolult parancssor kimenetét. 
\begin{table}[!h]
\begin{center}
\small
\begin{tabular}{|c|l|}
\hline
\verb@"..."@		&	változó behelyettesítés van\\
\verb@'...'@		&	változó behelyettesítés nincs\\
\verb@'"..."'@		& 	a \verb.". jelek elvesztik különleges jelentésüket\\
\verb@"'...'"@		&	a \verb.'. jelek elvesztik különleges jelentésüket\\
\verb@`...`@		& 	A közrezárt szöveg végrehajtásra kerül\\
\hline
\end{tabular}
\end{center}
\caption{Idézőjelek fajtái és használatuk} \label{table:quote}
\end{table}

%Futtassuk az alábbi szkriptet!

\begin{lstlisting}[title=scriptek/bash/behelyettesites.sh]
! /bin/bash
#NEV=Gizi
echo -n "Kerek egy nevet: "
read NEV
echo Nev: $NEV
echo "Nev: $NEV"
echo 'Nev: $NEV'
echo "Nev: '$NEV'"
echo '"Nev: $NEV"'
\end{lstlisting}

\emph{Eredmény (Ha a NEV értéke Gizi)}
\begin{lstlisting}
Nev: Gizi
Nev: Gizi
Nev: $NEV
Nev: 'Gizi'
"Nev: $NEV"
\end{lstlisting}



\subsubsection*{A beágyazott utasítás}\label{beagyazott}
\begin{lstlisting}
...$(utasitas)...
\end{lstlisting}
Beágyazott utasítást bárhová elhelyezhetünk, ahol változó értékét behelyettesíthetjük. A
beágyazott utasítás is behelyettesítés. A BASH a \verb@$()@ kifejezésen belül található 
utasítást parancsként (programként) végrehajtja. A kifejezés behelyettesítési értéke a 
program szabványos kimenetén megjelenő lista lesz.
%
A beágyazott utasítás egy formája a visszafele hajló idézőjel, lásd a(z) \ref{table:quote}.~táblázatot 
(\pageref{table:quote}.~oldal). Tehát a(z) \verb.$(utasitas). ekvivalens a(z) \verb.`utasitas`.-al.



\subsection{A környezeti változók}

A környezeti változók névvel és értékkel rendelkező eszközök, amelyek nevüket onnan kapták, hogy a munkakörnyezetet írják le a programok számára. Minden folyamat rendelkezik környezeti változókkal, nemcsak a BASH, nem csak a héjprogramok. 
(Ahogyan minden folyamat rendelkezik munkakönyvtárral is!)
%\smallskip


\begin{table}[!h]
\centering\small
 \begin{tabular}{|ll|}
\hline
  \tt\$BASH	& A futtató bináris állomány
  \\
  \tt\$HOME	& A felhasználó saját könyvtára
  \\
  \tt\$USER	& Felhasználó login neve
  \\
\tt\$USERNAME	& A felhasználó teljes neve
  \\
  \tt\$HOSTNAME	& A gép neve
  \\
\tt\$PWD		& Aktuális könyvtár
  \\
  \tt\$PATH	& A parancsok keresési útja
  \\
  \tt\$COLUMNS	& Betűoszlopok száma a képernyőn
  \\
\tt\$LINES		& Sorok száma  képernyőn
  \\
\tt\$TERM		& Terminál típusa
  \\
\tt\$EDITOR	& Alapértelmezés szerinti szövegszerkesztő
  \\
\hline
 \end{tabular}
\caption{Fontosabb környezeti változók}
\end{table}
\clearpage

A környezeti változók kezelése egyirányú:
\begin{itemize}
\item A szülőfolyamat meghatározza a gyermekfolyamat környezeti változóit, de a
gyermek nem változtathatja meg a szülő környezeti változóit.
\item  Amikor egy folyamat egy másik folyamatot indít, másolat készül a környezeti
változóiról.
\item A folyamat futása közben használhatja, megváltoztathatja a környezeti változóit.
\item A folyamat futása közben újabb környezeti változókat hozhat létre.
\item Amikor egy folyamat befejeződik, a környezeti változói megsemmisülnek.
\end{itemize}



\subsubsection*{A környezeti változók használata}

A BASH programban a környezeti változó értékét ugyanúgy helyettesíthetjük be, mint a BASH saját változót.
\begin{lstlisting}
#!/bin/bash

echo $USER@$HOSTNAME
\end{lstlisting}
Ezeket a változókat a BASH program a szülőfolyamattól örökölte.


\subsubsection*{A környezeti változó megváltoztatása}
Ha megváltoztatjuk egy környezeti változó értékét, az utána indított programok már az új értékét kapják meg. (A BASH változók értékét azonban nem kapják meg az indított programok.)

\begin{lstlisting}
#!/bin/bash

USER=gizi
firefox
\end{lstlisting}

Ha a környezeti változó létezett, az új értéket kapják meg a gyermekfolyamatok; 
  ha nem, akkor csak egy BASH változót hoztunk létre.

\subsubsection*{Környezeti változó létrehozása}
Környezeti változót a BASH változóból, az \texttt{export} parancs segítségével hozhatunk létre.
\begin{lstlisting}
#!/bin/bash
IZE="mintamokus"
export IZE
\end{lstlisting}

Az \texttt{export} parancs kiadása után több változónevét is írhatjuk, mindegyikből környezeti változó lesz. Az egyszerűsített írásmód esetén az értékadást és az \texttt{export} kulcsszót egy sorban helyezzük el. Az export parancs kiadása után több értékadást is írhatuk, mindegyik környezeti változót hoz létre.
\begin{lstlisting} 
#!/bin/bash
export IZE="mintamokus"
\end{lstlisting}




\subsection{Parancssori paraméterek}
A héjprogramok meghívásakor átadhatunk egy vagy több paramétert. Több paraméter esetén azokat egy vagy több szóköznek  kell elválasztania. Ha az átadandó paraméter maga is tartalmaz szóközt, kettős idézőjelbe kell tenni. 
%
A parancssori paraméterek értékére a \verb.$1,$2.,\dots szimbólumokkal hivatkozhatunk. A szám a kérdéses paraméter sorszáma. A 0 sorszámú paraméter minden esetben maga a meghívott program neve. 
\begin{table}[h!]
\begin{center}
\small
\begin{tabular}{|cl|}
\hline
\verb.$#.	& A parancssori paraméterek száma\\
\verb.$?.	& A legutoljára végrehajtott parancs visszatérési értéke\\
\verb.$$.	& A futó program folyamatazonosítója\\
\verb.$n.	& Az $n$-edik parancssori paraméter értéke\\
\verb.$0.	& A pillanatnyi héjprogram neve\\
\verb.$*.	& Valamennyi parancssori paraméter egyben, egyetlen karakterláncként (\verb@"$1 $2 ... $9 ..."@)\\
\verb.$@.	& Valamennyi parancssori paraméter egyben, egyenként idézőjelbe téve (\verb@"$1" "$2" ... "$9" ...@)\\
\hline
\end{tabular}
\end{center}
\caption{A héj névvel nem rendelkező belső változói}
\end{table}

\subsection{Matematikai kifejezések}

Számos esetben szükségünk lehet egyszerű matematikai műveletekre a héjprogramozás során. 


Az \texttt{expr} egy négy alapműveletes számológép, de kizárólag egész számokkal képes műveleteket végezni. 
\begin{table}[h!]
\centering\small
\begin{tabular}{|ll|}
\hline
\verb.|.	& Logikai VAGY operátor. Visszatérési értéke az első paraméter, \\
		& ha az nem nulla vagy nem üres karakterlánc, egyébként a második.
\\
\hline
\verb.&.	& Logikai ÉS operátor. Visszatérési értéke az első paraméter, \\
		& ha egyik argumentuma sem nulla vagy üres karakterlánc. 
			Ellenkező esetben nullát ad vissza.\\
\hline
\end{tabular}
\caption{Az \texttt{expr} logikai operátorai}
\end{table}


\begin{table}[h!]
\centering\small
\begin{tabular}{|ll|}
\hline
\tt<	& Kisebb\\
\tt<=	& Kisebb vagy egyenlő\\
\tt> 	& Nagyobb\\
\tt>=	& Nagyobb vagy egyenlő\\
\tt=, ==	& Egyenlő\\
\tt!=	& Nem egyenlő\\
\hline
\end{tabular}
\caption{Az \texttt{expr} által ismert relációs jelek} \label{table:rel2}
\end{table}


\begin{table}[h!]
\centering\small
\begin{tabular}{|cl|}
\hline
\verb.+.	& Összeadás\\
\verb.-.	& Kivonás\\
\verb.*.	& Szorzás\\
\verb./.	& Osztás\\
\verb.%.	& Maradékképzés\\
\hline
\end{tabular}
\caption{Az \texttt{expr} műveletei} \label{table:rel}
\end{table}

A relációs jelek (\ref{table:rel}. táblázat) használatakor az \texttt{expr} 1-et ad vissza, ha az összehasonlítás igaz, nullát ha hamis. Az összehasonlítás elvégzése előtt megkísérli számokká alakítani a megadott paramétereket. Ha ez sikeres, aritmetikai összehasonlítást végez. Ha bármelyik paramétert nem képes átalakítani, akkor az összehasonlítás betűrend szerinti (lexikografikus) lesz. Ilyenkor az a paraméter számít nagyobbnak, amelyikben előbb következik magasabb ASCII kódú karakter.

Aritmetikai műveletek csak számokon hajthatóak végre, így ha valamelyik paraméter nem alakítható számmá, hiba keletkezik. 

\emph{Buktató:} ügyeljünk arra, hogy az \texttt{expr} az egyszerű műveleti jeleket is csak akkor hajlandó értelmezni, ha azokat szóközök választják el a tényezőktól. Így a 
\verb.SZAM=`expr 3+2`.
forma például helytelen. A helyes írásmód:
\begin{verbatim}
SZAM=`expr 3 + 2`
\end{verbatim}

Arra is figyelni kell, hogy az \texttt{expr} egyes műveletei a héj számára is értelmesek és ha elfelejtjük 
levédeni ezeket a \verb.\. karakterrel, furcsa mellékhatásokat tapasztalhatunk. 

\subsubsection*{Példa az expr használatára}
\texttt{scriptek/matematika/expr.sh}
%\bigskip

Nem csak az \texttt{expr} használható matematikai műveletek elvégzésére, hanem a \texttt{let} parancs is.
%\bigskip

\begin{lstlisting}[title=Példa let használatára (scriptek/matematika/let.sh)]
#! /bin/bash
COUNTER=10
echo "$COUNTER"

let COUNTER-=1
echo "$COUNTER"

let COUNTER=COUNTER-1
echo "$COUNTER"


echo "$((COUNTER-=1))"
echo "$((COUNTER=COUNTER-1))"

echo "$[COUNTER-=1]"
echo "$[COUNTER=COUNTER-1]"
\end{lstlisting}



\subsection{for}

\begin{minipage}{0.4\textwidth}
\lstset{linewidth=\textwidth}
\begin{lstlisting}
for valtozo in lista; do
	utasitasblokk
done
\end{lstlisting}
\end{minipage}
\hfill
\begin{minipage}{0.4\textwidth}\lstset{linewidth=\textwidth}
\begin{lstlisting}
for valtozo in lista
do
	utasitasblokk
done
\end{lstlisting}
\end{minipage}


A változó rendre felveszi a lista elemeinek értékét és minden értékkel végrehajtódik az
utasítás blokk minden utasítása.\smallskip

\emph{A változó lehet:}
\begin{itemize}
\item BASH változó neve, vagy
\item Környezeti változó neve.
\end{itemize}

A változó neve elé nem kell \$ jelet írnunk!  Az utasításblokk tetszőlegesen sok utasításból állhat, 
amelyekben használhatjuk a változó értékét. A változó neve elé ilyenkor \$ jelet kell írnunk.
\smallskip

\emph{A lista lehet:}
\begin{itemize}
\item Szavak szóközökkel elválasztott listája, vagy
\item Állománynév helyettesítő karakterekkel megadott állománylista, vagy
\item változók listája, vagy
\item végrehajtandó parancs a \verb@$(...)@  vagy \verb@`...`@ szerkezettel.
\end{itemize}
A listát lezárja:
\begin{itemize}
\item a pontosvessző, vagy
\item az újsor karakter.
\end{itemize}


\begin{lstlisting}
#!/bin/bash

for DAY in hetfo kedd szerda; do
  echo $DAY
  mkdir $DAY
done
\end{lstlisting}
% A for ciklus példa 2 - for/pelda-2.sh
\begin{lstlisting}
#!/bin/bash

for FILE in *.dvi; do
  echo "Nyomtatas: $FILE"
  dvips $FILE
done
\end{lstlisting}

\begin{lstlisting}
#!/bin/bash

MENTENI="$HOME/bin $HOME/munka"
for KONYVTAR in $MENTENI;do
  echo "Mentes: $KONYVTAR"
  cp -r $KONYVTAR /mnt/pendrive
done
\end{lstlisting}

\begin{lstlisting}
#!/bin/bash

for ORA in $(seq 1 24);do
  echo "Letrehoz: $ORA"
  mkdir /mnt/pendrive/$ORA
done
\end{lstlisting}

Beágyazott utasítást (lásd \pageref{beagyazott}. oldal) és for ciklust hatékonyan használhatunk együtt. 
A ciklus egyenként végigjárja az utasítás kimenetén megjelenő listát.
\begin{lstlisting}
#!/bin/bash

for F in $(ls -l | grep ^d | awk '{print $9}'); do
  echo "$F archivalasa"
  tar -czf $F.tar.gz $F
done
\end{lstlisting}

Beágyazott utasítás segítségével változók értékét is beállíthatjuk:
\begin{lstlisting}
#!/bin/bash

for F in *; do
  KISBETUS=$(echo $F | tr A-Z a-z)
  mv $F $KISBETUS
done
\end{lstlisting}

\subsubsection*{Egymásba ágyazott for ciklusok}
\begin{lstlisting}
for valtozo1 in lista1; do
  for valtozo2 in lista2; do
    utasitasblokk
  done
done
\end{lstlisting}\bigskip

\begin{lstlisting}
#!/bin/bash

for EMAIL in $(cat ~/cimek); do
    echo -n "Kuldes $EMAIL cimre:"
    for FILE in ~/LEVELEK/*.xt; do
        echo -n "$FILE"
        mail $EMAIL <$FILE
    done
    echo "[OK]"
done
\end{lstlisting}




\subsection*{Gyakorlásképp}
\begin{enumerate}
\item Írj szkriptet, mely beolvas egy szöveget, eltárolja azt, majd kiírja a konzolra.
\item Írj szkriptet, mely kiírja a paraméterként kapott fájlok típusát és tartalmát.
\end{enumerate}



