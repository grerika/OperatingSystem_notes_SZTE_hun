\hyphenation{il-lesz-ke-dik}
\begin{center}
\Ovalbox{\large \verb.awk 'PROGRAM' ÁLLOMÁNY(OK).}
\end{center}

\begin{itemize}
\item Leírás: \texttt{man awk/gawk, info gawk}
\item Mintakereső és -feldolgozó program saját programozási nyelvvel
\item Sorban beolvassa a bemeneti állomány(ok) tartalmát, miközben az AWK nyelven írt PROGRAM-ban leírt műveleteket végrehajtja. Szintén szűrő.
\item Ha nem adunk meg állományt, akkor a szabványos bemenetről olvas.
\item A forrásprogram szövegét érdemes aposztrófok közé zárni, hogy a benne szereplő karaktereket a shell ne tekintse speciálisnak.
\item \textbf{-f PROGFÁJL}: a végrehajtandó programot PROGFÁJL-ból olvassa
\end{itemize}

\noindent\Ovalbox{gawk} Az eredeti awk program GNU változata, GNU/Linux alatt ezt használhatjuk. % Jóval többet tud elődjénél.
\bigskip

\noindent\verb.#! /usr/bin/awk -f.: Ha az AWK forrásprogramot állományban tároljuk el, az
állomány első sorába ezt a megjegyzést (parancsértelmező fejlécet) írjuk, valamint
futtathatóvá tesszük az állományt, akkor az AWK prog\-ra\-mot a shell scriptek
mintájára a \verb@./PROGFÁJL ÁLLOMÁNY(OK)@ paranccsal is lefuttathatjuk.

\subsection{Forrásprogram felépítése}

Minden AWK forrásprogram \textbf{szabályok} sorozata. Minden szabály
tartalmazhat egy \textbf{mintát} és egy hozzá tartozó tevékenységet avagy
\textbf{akciót}. Az akciót különféle utasításokból állíthatjuk össze.

\begin{itemize}
\item A szabályok alakja: \verb.MINTA{AKCIÓ}.
\item A szabályokat egymástól sortöréssel vagy pontosvesszővel lehet elválasztani.
\item A feldolgozás során a bemenet tartalmát \emph{rekordok}ra (record) bontja, ezek
alapesetben a bemenet sorai lesznek. A rekordokat szintén továbbontja \emph{mezők}re
(field), amiket alapesetben az illető sor szavai képviselnek.
\item A bemenet feldolgozása rekordonként történik. Minden rekordot megpróbál
illeszteni sorban az összes szabály mintájára, az első szabálytól kezdve. Ha a
rekord illeszkedett egy szabály mintájára, akkor végrehajtódik a hozzá tartozó
akció. Végül az összes szabály ellenőrzése után rátér a következő rekord feldolgozására.
\item A szabályok sorrendje fontos, hiszen a mintákra való illeszkedés ellenőrzése, s
így az akciók végrehajtásának sorrendje ettől függ!
Hiányzó minta esetén az illető akció minden rekord esetén lefut.
\item A szabályokból az akciót is el lehet hagyni a kapcsos zárójelekkel együtt. A
hiányzó akció ekvivalens a \verb.{print}. akcióval, ami kiírja az egész rekord tartalmát.
\item Vigyázat! A \verb.{}. páros az üres akciót jelöli, tehát nem egyezik meg az előbb
említett esettel (ti. az akció elhagyásával)!
\item Bármely mintát vagy utasítást folytathatjuk a következő sorban, ha az
aktuális sort a $\backslash$ jellel zárjuk.
\item Az akciók utasításlistája akár több sorból is állhat. Egy sorba több utasítást
is írhatunk, ha őket pontosvesszővel (;) választjuk el egymástól. Hasonlóan,
a pontosvessző használatával több szabályt is írhatunk egy sorba.
\item Szóközöket és tabulátorokat tetszés szerint használhatunk a műveleti jelek,
operandusok, utasítások, paraméterek, stb. között. Üres sorok szintén megengedettek.
\item \#: A sor végéig tartó \emph{megjegyzés} kezdetét jelzi.
\item Az AWK is különbséget tesz a kisbetűk és nagybetűk között!
\end{itemize}



\subsection{Minták}

Minden minta egy logikai feltételt fogalmaz meg. Ha a feltétel teljesül egy
konkrét rekord esetén, akkor azt mondjuk, hogy a rekord illeszkedik a
mintára. Fontos, hogy olyan feltételt is megfogalmazhatunk, amely nem
(vagy nemcsak) a rekord tartalmától függ, hanem például valamely változótól!

\subsubsection*{Elemi minták}
\begin{description}
\item[\texttt{(MINTA)}] csoportosítás (műveleti sorrend felülbírálása), MINTÁ-ra
illeszkedik
\item[\texttt{!MINTA}] logikai tagadás (negáció)
\item[\texttt{/REGKIF/}] igaz, ha az egész rekord illeszkedik a reguláris kifejezésre
\item[\texttt{KIF$\sim$/REGEX/}] igaz, ha a KIF kifejezés  mint szöveg
illeszkedik a reguláris kifejezésre
\item[\texttt{KIF!$\sim$/REGEX/}] igaz, ha a kifejezés nem illeszkedik a REGKIF-re
\item[relációs kifejezések] tetszőleges kifejezés, amely relációs jelet tartalmaz
\item[\texttt{BEGIN}] csak a bemenet feldolgozása előtt teljesül\label{begin}
\item[\texttt{END}] csak a bemenet feldolgozása után teljesül
\end{description}

\begin{itemize}
\item A \texttt{BEGIN}  és \texttt{END}  mintákhoz mindig meg kell adni az akciót is! Továbbá
ezek a speciális minták nem kombinálhatók semmilyen más mintával,
valamint nem alkalmazható rájuk a csoportosítás és a negáció sem!
\item A \texttt{BEGIN}  mintához tartozó akció pontosan egyszer hajtódik végre,
mégpedig a legelső bemeneti rekord feldolgozása előtt. Ez akkor is így
történik, ha több bemeneti állományt adtunk meg.
\item Hasonlóan, az \texttt{END}  mintához tartozó akció is pontosan egyszer, az utolsó
bemeneti rekord feldolgozása után hajtódik végre. Ezt az awk program
befejeződése követi.
\end{itemize}


\subsubsection*{Összetett minták}
\begin{description}
\item[\texttt{MINTA1\&\&MINTA2}] logikai ÉS (konjunkció)
\item[\texttt{MINTA1||MINTA2}] logikai MEGENGEDŐ VAGY (diszjunkció)
\item[\texttt{MINTA1,MINTA2}] Rekordok tartományára illeszkedik, kezdve egy olyan
rekorddal, amely MINTA1-re illeszkedik, egészen egy olyan rekordig,
amely MINTA2-re illeszkedik. Nem kombinálható semmilyen más
mintával!
\end{description}


\subsection{Konstansok}
\begin{description}
\item[Szám avagy numerikus konstansok]\hfill
	\begin{itemize}
	\item egész számok (pl. 12)
	\item valós törtszámok tizedesponttal (pl. 25.3)
	\item egész vagy valós szám hatványkitevővel (pl. 1.234e+2=123.4)
	\end{itemize}
\item[Szöveges avagy sztring konstansok]\hfill
	\begin{itemize}
	\item \verb."SZOVEG".
	\item \verb."".: üres sztring (0 karakter hosszúságú szöveg)
	\item A szövegben a \verb.\. speciális (az ún. escape-karakter), így használhatók pl.
	a következő escape-szekvenciák: \verb.\\. (közönséges \verb.\.), \verb.\". (közönséges
	idézőjel), \verb.\n. (sortörés), \verb.\t. (tabulátor).
	\end{itemize}
\item[Konstans reguláris kifejezések]\hfill
	\begin{itemize}
	\item \texttt{/REGKIF/}
	\item A reguláris kifejezésen belül a \verb.\. speciális, így használhatók a \verb.\\.
	(közönséges \verb.\.) és \verb.\/. (közönséges \verb./.) karakterpárosok.
	\end{itemize}
\end{description}

\subsection{Változók}
\begin{itemize}
\item A változók \emph{élettartama dinamikus}, az első használatkor automatikusan létrejönnek (nem kell őket deklarálni).
\item A változók neve betűket, számokat és aláhúzásjelet (\verb._.) tartalmazhat, és nem kezdődhet számjeggyel.
\item \textbf{Változók típusai}
	\begin{itemize}
	\item  numerikus változók (valós számokat tárolnak)
	\item  szöveges változók avagy sztringek (string)
	\item  egydimenziós tömbök 	
	\end{itemize}

\item A tömböket kivéve minden változó \emph{típusa dinamikus}, azaz a használattól függően változik! \\
	Ez a tömbelemekre is vonatkozik 

\item Egy változó típusát nem lehet tömbről numerikusra vagy sztringre változtatni, és viszont!
\item A változók értékét az awk automatikusan konvertálja számmá vagy szöveggé, szintén a használati módtól 
(művelettől, függvénytől) függően. Ha a szöveget nem lehet számmá konvertálni 
(mert nem egy érvényes alakú számot tartalmaz), nullát kapunk.

\item \textbf{Manuális konverzió}\hfill
	\begin{itemize}
	\item szövegből szám: adjunk hozzá 0-t
	\item számból szöveg: fűzzük hozzá az üres sztringet (\verb."".)
	\end{itemize}
		
\item \texttt{NÉV=ÉRTÉK}\hfill
	\begin{itemize}
	\item Értékadás egy létező változónak, vagy új változó létrehozása. A változó típusa \texttt{ÉRTÉK} típusa lesz.
	\item A C programozási nyelv egyéb értékadó, növelő és csökkentő műveletei is használhatók
	\item Az ÉRTÉK természetesen nemcsak konstans, hanem kifejezés is lehet.
	\item Többszörös értékadás (\texttt{NÉV1=NÉV2=ÉRTÉK}) is megengedett.
	\end{itemize}
\item \texttt{NÉV}\hfill
	\begin{itemize}
	\item a változó aktuális értékét jelöli
	\item definiálatlan (ti. amelyiknek eddig nem adtunk értéket) változó értéke az üres sztring (\verb."".) ill. 0.
	\end{itemize}
\end{itemize}

\subsection{Beépített változók}
Az awk program indulásakor már létezik jónéhány különleges, \textbf{beépített változó} (built-in variable). 
Ezek neve egységesen csupa nagybetűből áll, és tartalmuk egyrészt a felhasználónak szóló fontos információkat hordoz,
másrészt némelyikük az awk program működését ill. a bemenet feldolgozásának módját vezérli.
\begin{description}
\item[\tt FILENAME] Az aktuális bemeneti állomány neve, illetve \texttt{-} a szabványos bemenet esetén. 

	A \texttt{BEGIN} minta (lásd \pageref{begin}. oldal) akcióján belül definiálatlan.
\item[\tt FNR] az aktuális rekord sorszáma az aktuális bemeneti állományon belül
\item[\tt FS] bemeneti mezőhatároló karakter (input field separator), kezdetben a szóköz
\item[\tt IGNORECASE] Ha értéke nemzérus, akkor a sztringek összehasonlítása ill. a
reguláris kifejezések illesztése nem különbözteti meg a kisbetűket a nagyoktól. Alapesetben értéke definiálatlan (effektíve nulla).
\item[\tt NF] az aktuális rekord mezőinek száma (number of fields)
\item[\tt NR] Az aktuális rekord sorszáma az eddig feldolgozott bemenet tekintetében. 

	Egy bemeneti állomány ill. a szabványos bemenet esetén egyenlő az \texttt{FNR}-rel.
\item[\tt OFS] Kimeneti mezőhatároló (output field separator), kezdetben a szóköz. 

	Értéke tetszőleges szöveg lehet, nemcsak egy karakter.
\item[\tt ORS] Kimeneti rekordhatároló (output record separator) kezdetben a sortörés. 

	Ez is tetszőleges szöveget tartalmazhat.
\item[\tt RS] bemeneti rekordhatároló karakter (input record separator), kezdetben a sortörés
\end{description}



\subsection{Mezők}
A bemenet rekordokra bontását, ill. azoknak mezőkre bontását két beépített változó vezérli. 
Az \texttt{RS} változó tartalma egy karakter (alapesetben sortörés), ez jelzi a rekordokat elválasztó karaktert. 
Hasonlóan, az \texttt{FS} változó tartalma (alapesetben szóköz) határozza meg, mi határolja a mezőket a rekordokon
belül. Ha az \texttt{FS} értéke a szóköz (alapeset), akkor a mezőket legalább egy szóköz vagy tabulátor választja el.

\begin{itemize} 
\item  Az aktuális rekord mezőinek a számát az \texttt{NF} beépített változó tárolja.
\item  A mezők típusa ugyancsak numerikus vagy szöveges lehet, az aktuális használattól függően. Összehasonlításkor a mezők tartalmát számnak tekinti az awk, ha az valóban egy érvényes számot tartalmaz, továbbá ha a másik tag szám konstans, numerikus változó vagy mezőhivatkozás.
\item \verb.$KIF.\hfill
	\begin{itemize}
	\item  Az aktuális rekord megadott sorszámú mezőjének tartalma. Ezt a jelölést mezőhivatkozásnak nevezzük.
	\item  Tetszőleges kifejezést is használhatunk, például \verb.$(2*3). a hatodik mezőt jelzi. 
			Természetesen a negatív értékek nem megengedettek.
	\item  \verb.$NF. az aktuális rekord utolsó mezőjének tartalma
	\item  \verb.$0. (dollárjel és nulla): az aktuális rekord teljes tartalma
	\end{itemize}
\item  \verb.$KIF=ÉRTÉK.\hfill
	\begin{itemize}
	\item egy adott mező – ill. \verb.KIF = 0., esetén a rekord – értékének módosítása
	\item Ha \verb.$0. tartalmát változtatjuk meg. akkor minden mező új értéket kap.
	Ha viszont egy mező tartalmát módosítjuk, akkor \verb.$0. értékét az awk
		újraépíti oly módon, hogy a mezőket az \texttt{OFS} értéke határolja majd el.
	\item Ha \texttt{KIF > NF}, akkor a mezők számát kibővíti, és \texttt{NF}-et is módosítja.
	Szükség szerint a közbülső helyekre új mezőket szúr be, ezek értéke az
		üres sztring (\verb."".) lesz. Végül pedig \verb.$0. tartalmát is újraszámítja az előbb leírt módon.
	\end{itemize}
\end{itemize}



\subsection*{Gyakorló feladatok}
Dolgozzuk fel az \texttt{ls -l} parancs kimenetét egy tetszőlegesen választott könyvtár esetén!

\begin{enumerate}
\item Írjuk ki csak a könyvtárbejegyzések nevét és méretét! Vegyük figyelembe, hogy névben előfordulhat szóköz is, 
	tehát a  név nem minden esetben a 8. mező maga! Emellett a linkeket, mint speciális eseteket kezelni kell, 
	hiszen a linkeknél a 8. mezőtől kezdve az utolsóig nem csak a link nevét, hanem magát a célfájl útvonalát is tartalmazza	 
	\verb.linknev -> celfajl. formában. 
	Segíthet a megoldásban a \pageref{stringawk}. oldalon található \textit{Szöveges függvények} alfejezet.

\item Számoljuk meg hány link, hány közönséges fájl és hány könyvtár szerepel a parancs kimenetén!

\item Összegezzük a fájlok méretét és vessük össze egyezik-e az eredmény azzal, amit kapunk a \texttt{du -s} paranccsal!

\item Állapítsuk meg melyik a legkisebb / legnagyobb fájl (a nevét és a méretét is írjuk ki), és számoljunk átlagos fájl méretet!
\end{enumerate}


+1 Dolgozzunk fel egy szöveges fájlt, ami egy mátrixot tárol, ahol az elemeket egy sorban pontos vessző választja el. 
%
%{
%\hspace{10mm}\small
%\begin{verbatim}
%7747;16811;29719;13280
%17718;5624;2746;21719
%32436;2799;22728;16972
%\end{verbatim}
%}
%
\textit{Ötlet:} generáláshoz használjuk a bash-t és a RANDOM változó értékét! 
%A RANDOM változóval véletlen egészt kapunk, tehát a
%\begin{small}
%\begin{verbatim}
%#! /bin/bash
%
%for ((i=0;i<$1;i++)); do
%    for ((j=0;j<$2-1;j++)); do
%          echo -n $RANDOM";"
%     done
%     echo $RANDOM
%done
%\end{verbatim}
%\end{small}
%generál nekünk egy mátrixot, ha pozitív egészt kap parancssori paraméterként.

\begin{enumerate}
\setcounter{enumi}{4}
\item Számoljuk sorösszeget és -átlagot minden rekordra és oszlopösszeget és -átlagot minden oszlopra. 
\end{enumerate}
