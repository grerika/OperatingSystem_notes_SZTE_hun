Forrás: \url{http://www.cyberciti.biz/faq/bash-for-loop-array/}


\subsubsection*{Deklarálás}


\begin{lstlisting}
declare -a arrayname
\end{lstlisting}

vagy

\begin{lstlisting}
array=( one two three )
files=( "/etc/passwd" "/etc/group" "/etc/hosts" )
limits=( 10, 20, 26, 39, 48)
\end{lstlisting}

vagy

\begin{lstlisting}
array[1]=one
array[2]=two
array[4]=three
\end{lstlisting}

A tömbök indexei egészek lehetnek, nem feltétlenül folytonosak.

\subsubsection*{Kiiratás, használat}
A \texttt{printf} használatával kiirathatjuk a tartalmunkat, például így
\begin{lstlisting}
printf "%s\n" "${array[@]}"
printf "%s\n" "${files[@]}"
printf "%s\n" "${limits[@]}"
\end{lstlisting}

vagy számláló ciklussal végigjárhatjuk a tömb elemeit

\begin{lstlisting}
for i in "${tombnev[@]}"
do
	echo $i
done
\end{lstlisting}


Ha az indexek értékeire vagyunk kíváncsiak, azt is lekérhetjük, azt a \verb.${!tombnev[*]}. tartalmazza

Ha szeretnénk kilistázni az indexeket a hozzájuk tartozó tömbértékekkel, akkor azt így tehetjük:
\begin{lstlisting}
for I in ${!array[*]}; do
  echo $I: ${array[$I]}
done
\end{lstlisting}

%\dots és még egy példa arra, hogy a tömbelemek típusa vegyes is lehet
\begin{lstlisting}

array=( "/etc/passwd" "/etc/group" "/etc/hosts" )
array[5]=5
array[7]=7
array[8]="nyolc"
array[100]="szaz"


for I in ${!array[*]}; do
  echo -e "$I\t${array[$I]}"
done
\end{lstlisting}
